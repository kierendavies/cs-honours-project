% !TEX root = paper.tex
\documentclass[paper.tex]{subfiles}

\begin{document}

\section{Code for reasoner methods}
\label{app:code}

\newlength{\currentparindent}
\setlength{\currentparindent}{\parindent}
\noindent
\begin{minipage}[t]{\dimexpr.5\textwidth-.5\columnsep}
\setlength{\parindent}{\currentparindent}

Here we present example code snippets corresponding to the methods introduced in section \ref{sec:reasoners}.  All code is in Java and uses the OWL API version 5.0.  The following imports and declarations apply as boilerplate for all snippets.

\end{minipage}

\bigskip

\noindent
\begin{verbatim}
import java.util.Stream;
import org.semanticweb.owlapi.model.*;
import org.semanticweb.owlapi.reasoner.OWLReasoner;

OWLReasoner reasoner;
OWLClassExpression C;
OWLNamedIndividual a;
\end{verbatim}

\bigskip

\makeatletter
\newcolumntype{V}[1]{>{\topsep=0pt\@minipagetrue}p{#1}<{\vspace{-\baselineskip}}}
\makeatother
\noindent
\renewcommand{\arraystretch}{1.5}
\begin{tabular}{p{5cm}V{11.9cm}} \hline
Method call & Code \\ \hline

$\textproc{isSatisfiable}(C)$ &
\begin{verbatim}
reasoner.isSatisfiable(C)
\end{verbatim}
\\

$\textproc{getSubClasses}(C)$ &
\begin{verbatim}
Stream.concat(
  reasoner.getSubClasses(C).entities(),
  reasoner.getEquivalentClasses(C).entities()
)
\end{verbatim}
\\

$\textproc{getInstances}(C)$ &
\begin{verbatim}
reasoner.getInstances(C).entities()
\end{verbatim}
\\

$\textproc{getSameIndividuals}(a)$ &
\begin{verbatim}
reasoner.getSameIndividuals(a).entities()
\end{verbatim}
\\

$\textproc{getDifferentIndividuals}(a)$ &
\begin{verbatim}
reasoner.getDifferentIndividuals(a).entities()
\end{verbatim}
\\

\hline
\end{tabular}

\end{document}
