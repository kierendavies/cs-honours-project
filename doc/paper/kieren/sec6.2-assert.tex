% !TEX root = paper.tex
\documentclass[paper.tex]{subfiles}

\begin{document}
\subsection{Assertions}
\label{sec:algorithms:assert}

Begin by noting that the addition of an ABox axiom does not affect satisfiability of classes, so $O$ cannot become incoherent.  We take this as given, without proof, for all axioms tested in this section.

Throughout this section, when an algorithm accepts $n$ individuals as arguments, we use the shorthand
\[\alpha = \{a_1, \ldots, a_n\}\]

\todo[connect, ref]
\todo[notation]

\begin{algorithm}[H]
  \caption{test $a_1 \equiv \ldots \equiv a_n$}
  \begin{algorithmic}[1]
    \raggedright
    \Input{
      $a_1, \ldots, a_n$ individuals \\
      $n \ge 2$
    }
    \Function{testSameIndividual}{$a_1, \ldots, a_n$}
      \If{$\{a_2, \ldots, a_n\} \subseteq \Call{getSameIndividuals}{a_1}$}
        \State \Return $\entailed$
        \label{alg:testSameIndividual:returnEntailed}
      \Else
        \For{$i \gets 1$ \To $n$}
          \State $\delta \gets \Call{getDifferentIndividuals}{a_i}$
          \If{$\{a_1, \ldots, a_n\} \intersect \delta \ne \emptyset$}
            \State \Return $\inconsistent$
            \label{alg:testSameIndividual:returnInconsistent}
          \EndIf
        \EndFor
        \State \Return $\absent$
      \EndIf
    \EndFunction
  \end{algorithmic}
\end{algorithm}

\begin{proposition}
  \label{prop:testSameIndividualEntailed}
  \sloppy~
  \textproc{testSameIndividual} is sound and complete with respect to entailment.
\end{proposition}
\begin{proof}
  Suppose the algorithm returns entailed.  This can only happen from line \ref{alg:testSameIndividual:returnEntailed}, inside the first if-condition, which is reached if and only if
  \begin{align*}
    & \{a_2, \ldots, a_n\} \subseteq \Call{getSameIndividuals}{a_1} \\
    \iff{}&(\forall a_j \in \{a_2, \ldots, a_n\}) \;\; a_j \in \Call{getSameIndividuals}{a_1} \\
    \iff{}& (\forall a_j \in \{a_2, \ldots, a_n\}) \;\; a_j \equiv a_1
  \end{align*}
  Equivalence of individuals is transitive, so this is the same as $a_1 \equiv \ldots \equiv a_n$ being entailed.  The chain of implication holds in both directions, giving both soundness and completeness.
\end{proof}

\begin{proposition}
  \textproc{testSameIndividual} is sound with respect to inconsistency.
\end{proposition}
\begin{proof}
  The algorithm can only return inconsistent at line \ref{alg:testSameIndividual:returnInconsistent}, so the first if-condition is false, and for some iteration of the for-loop the inner if-condition is true.  From the inner condition, we have
  \begin{align*}
    & (\exists i \le n) \;\; \alpha \intersect \Call{getDifferentIndividuals}{a_i} \ne \emptyset \\
    \iff{}& (\exists a_i, a_j \in \alpha) \;\; a_j \in \Call{getDifferentIndividuals}{a_i} \\
    \iff{}& (\exists a_i, a_j \in \alpha) \;\; O \sementails a_i \not\equiv a_j
  \end{align*}
  Which contradicts the axiom under test, and therefore
  \[ \test_O(a_1 \equiv \ldots \equiv a_n) = \inconsistent \qed \]
\end{proof}

\begin{proposition}
  \textproc{testSameIndividual} is complete with respect to inconsistency.
\end{proposition}
\begin{proof}
  Suppose the algorithm does not return inconsistent.  If it returns entailed, by proposition \ref{prop:testSameIndividualEntailed}, $\test_O(a_1 \equiv \ldots \equiv a_n) = \entailed$.  If not, then for every iteration of the for-loop the inner if-condition must be false.  That is,
  \[ (\forall a_i, a_j \in \alpha) \;\; O \nsementails a_i \not\equiv a_j \]
  It follows that $O$ has an interpretation in which $a_1 \equiv \ldots \equiv a_n$, so $O \union (a_1 \equiv \ldots \equiv a_n)$ is consistent.

  In both cases, $\test_O(a_1 \equiv \ldots \equiv a_n) \ne \inconsistent$, so by contrapositive the proposition holds.
\end{proof}

\begin{theorem}
  \label{prop:testSameIndividualCorrect}
  \textproc{testSameIndividual} is correct and terminating.
\end{theorem}
\begin{proof}
  The algorithm is sound and complete with respect to entailment and inconsistency, and by default also absence.  All loops are bounded.
\end{proof}

\todo[connect, ref]

\begin{algorithm}[H]
  \caption{test $\oaxiom{DifferentIndividuals}(a_1, \ldots, a_n)$}
  \begin{algorithmic}[1]
    \raggedright
    \Input{
      $a_1, \ldots, a_n$ individuals \\
      $n \ge 2$
    }
    \Function{testDifferentIndividuals}{$a_1, \ldots, a_n$}
      \For{$i \gets 1$ \To $n$}
        \State $\gamma \gets \Call{getSameIndividuals}{a_i}$
        \If{$(\{a_1, \ldots, a_n\} \setminus \{a_i\}) \intersect \gamma \ne \emptyset$}
          \State \Return $\inconsistent$
          \label{alg:testDifferentIndividuals:returnInconsistent}
        \EndIf
      \EndFor
      \For{$i \gets 1$ \To $n$}
        \State $\delta \gets \Call{getDifferentIndividuals}{a_i}$
        \If{$(\{a_1, \ldots, a_n\} \setminus \{a_i\}) \nsubseteq \delta$}
          \State \Return $\absent$
        \EndIf
      \EndFor
      \State \Return $\entailed$
      \label{alg:testDifferentIndividuals:returnEntailed}
    \EndFunction
  \end{algorithmic}
\end{algorithm}

\begin{proposition}
  \sloppy~
  \textproc{testDifferentIndividuals} is sound and complete with respect to entailment.
\end{proposition}
\begin{proof}
  The algorithm can only return entailed at line \ref{alg:testDifferentIndividuals:returnEntailed}.  To reach this line, the two if-conditions must be false for every iteration of their enclosing for-loops, else another result would be returned.  This means
  \begin{align*}
    & (\forall a_i \in \alpha) \;\; (\alpha \setminus \{a_i\}) \intersect \Call{getSameIndividuals}{a_i} = \emptyset \\
    {}\land{}& (\forall a_i \in \alpha) \;\; (\alpha \setminus \{a_i\}) \subseteq \Call{getDifferentIndividuals}{a_i}
  \end{align*}
  Equivalently, for all distinct arguments $a_i, a_j \in \alpha$ with $i \ne j$
  \begin{align*}
    & \begin{multlined}[t]
      a_j \notin \Call{getSameIndividuals}{a_i} \\
      {} \land a_j \in \Call{getDifferentIndividuals}{a_i}
    \end{multlined} \\
    \iff{}& O \nsementails a_i \equiv a_j \;\land\; O \sementails a_i \not\equiv a_j
  \end{align*}
  This holds if and only if the ontology already entails the axiom.
\end{proof}

\begin{proposition}
  \sloppy~
  \textproc{testDifferentIndividuals} is sound with respect to inconsistency.
\end{proposition}
\begin{proof}
  The algorithm can only return inconsistent at line \ref{alg:testDifferentIndividuals:returnInconsistent}.  So for some iteration of the first for-loop, its inner if-condition is true.  That is,
  \[ (\exists a_i \in \alpha) \;\; (\alpha \setminus \{a_i\}) \intersect \Call{getSameIndividuals}{a_i} \ne \emptyset \]
  So there exist distinct $a_i, a_j \in \alpha$ such that
  \begin{align*}
    & a_j \in \Call{getSameIndividuals}{a_i} \\
    \iff{}& O \sementails a_i \equiv a_j
  \end{align*}
  This contradicts the axiom under test, so the ontology would become inconsistent if the axiom were added.
\end{proof}

\begin{proposition}
  \sloppy~
  \textproc{testDifferentIndividuals} is complete with respect to inconsistency.
\end{proposition}
\begin{proof}
  Suppose the algorithm does not return inconsistent.  Then for every iteration of the first for-loop its inner if-condition must be false.  That is,
  \[ (\forall a_i \in \alpha) \;\; (\alpha \setminus \{a_i\}) \intersect \Call{getSameIndividuals}{a_i} = \emptyset \]
  So for all distinct $a_i, a_j \in \alpha$ we have $O \nsementails a_i \equiv a_j$.  There exists an interpretation of $O$ in which the individuals are all different, which then also models
  \[ O \union \oaxiom{DifferentIndividuals}(a_1, \ldots, a_n) \]
  Therefore ontology with the axiom added is not inconsistent, so by contrapositive the proposition holds.
\end{proof}

\begin{theorem}
  \textproc{testDifferentIndividuals} is correct and terminating.
\end{theorem}
\begin{proof}
  As for theorem \ref{prop:testSameIndividualCorrect}.
\end{proof}

\todo[connect, ref]

\begin{algorithm}[H]
  \caption{test $a : C$}
  \begin{algorithmic}[1]
    \raggedright
    \Input{
      $C$ class expression \\
      $a$ individual
    }
    \Function{testClassAssertion}{$C, a$}
      \If{$a \in \Call{getInstances}{C}$}
        \State \Return $\entailed$
        \label{alg:testClassAssertion:returnEntailed}
      \ElsIf{$a \in \Call{getInstances}{\neg C}$}
        \State \Return $\inconsistent$
        \label{alg:testClassAssertion:returnInconsistent}
      \Else
        \State \Return $\absent$
      \EndIf
    \EndFunction
  \end{algorithmic}
\end{algorithm}

\begin{proposition}
  \label{prop:testClassAssertionEntailed}
  \sloppy~
  \textproc{testClassAssertion} is sound and complete with respect to entailment.
\end{proposition}
\begin{proof}
  The algorithm can only return entailed at line \ref{alg:testClassAssertion:returnEntailed}.  This is reached if and only if the first if-condition is true.  That is,
  \begin{align*}
    & a \in \Call{getInstances}{C} \\
    \iff{}& O \sementails a : C \\
    \iff{}& \test_O(a : c) = \entailed \qed
  \end{align*}
\end{proof}

\begin{proposition}
  \sloppy~
  \textproc{testClassAssertion} is sound with respect to inconsistency.
\end{proposition}
\begin{proof}
  Suppose the algorithm returns inconsistent.  This can only happen at line \ref{alg:testClassAssertion:returnInconsistent}, so the first if-condition is false and the second is true.  That is,
  \begin{align*}
    & a \notin \Call{getInstances}{C} \land a \in \Call{getInstances}{\neg C} \\
    \iff{}& O \nsementails a : C \;\land\; O \sementails a : \neg C
  \end{align*}
  This contradicts $a : C$, so $O \union (a : C)$ is inconsistent.
\end{proof}

\begin{proposition}
  \textproc{testClassAssertion} is complete with respect to inconsistency.
\end{proposition}
\begin{proof}
  Suppose the algorithm does not return inconsistent.  If it returns entailed, then by proposition \ref{prop:testClassAssertionEntailed} we have that $\test_O(a : C) = \entailed$.  If not, then the second if-condition must be false.
  \begin{align*}
    & a \notin \Call{getInstances}{\neg C} \\
    \iff{}& O \nsementails a : \neg C
  \end{align*}
  So $O$ has an interpretation which also models $O \union (a : C)$ which is therefore consistent.  By contrapositive, the proposition holds.
\end{proof}

\begin{theorem}
  \textproc{testClassAssertion} is correct and terminating.
\end{theorem}
\begin{proof}
  As for theorem \ref{prop:testSameIndividualCorrect}.
\end{proof}

\end{document}
