% !TEX root = paper.tex
\documentclass[paper.tex]{subfiles}

\begin{document}

\section{Example usage}
\label{sec:examples}

Here we present some examples of axioms being tested to illustrate how the algorithms are used and how they work.
We use an ontology $O$ which consists of the following axioms.
\begin{gather*}
  \oaxiom{DisjointClasses}(\oent{Animal}, \oent{Plant}) \\
  \begin{aligned}
    \oent{Giraffe} &\sqsubseteq \oent{Mammal} \\
    \oent{Mammal} &\sqsubseteq \oent{Animal} \\
    \oent{Herbivore} &\equiv \forall \oent{eats} .\, \oent{Plant} \\
    \oent{Susan} &\aligncolon \oent{Giraffe} \\
    \oent{Frank} &\aligncolon \oent{Herbivore}
  \end{aligned}
\end{gather*}

Let us begin with a simple example which falls into use case \ref{enum:usecase:regression}, testing something we expect to be entailed to assure the quality of the ontology.
We test that $\oent{Giraffe}$ is a subclass of $\oent{Animal}$.
In terms of our formal model, this means finding the result of
\[ \test_O(\oent{Giraffe} \sqsubseteq \oent{Animal}) \]
In OWL 2 functional syntax, the axiom is represented as
\[ \oaxiom{SubClassOf}(\oent{Giraffe}, \oent{Animal}) \]
So we call the corresponding algorithm
\[ \Call{testSubClassOf}{\oent{Giraffe}, \oent{Animal}} \]
The algorithm first checks if there are any instances of the class expression $\oent{Giraffe} \sqcap \neg \oent{Animal}$.
There are none in this ontology, so it proceeds to check if the same class expression has any named subclasses or equivalent classes.
Again there are none, so it checks if the class expression is satisfiable.
It is not, so the algorithm returns entailed.

Next we consider some examples of axioms which are not entailed.
These could arise from mistakes in the ontology or in a test, or from use case \ref{enum:usecase:temp}, temporary tests used for investigation.

We test if $\exists \oent{eats} .\, \oent{Animal} \sqsubseteq \oent{Herbivore}$.
Similarly to the previous example, we call
\[ \Call{testSubClassOf}{\exists \oent{eats} .\, \oent{Animal}, \oent{Herbivore}} \]
This checks if $\exists \oent{eats} .\, \oent{Animal} \sqcap \neg \oent{Herbivore}$ has instances, which it does not, then named subclasses, which it does not.
Then it checks if it is satisfiable, which it is because the ontology does not entail that it is empty, so the algorithm returns absent.
This means that the axiom is not entailed but it would not cause inconsistency or incoherence if added to the ontology.
\todo[drop last sentence?]

Note that it is not possible to test this axiom with any of the tools identified in section \ref{sec:related} because the left part of the axiom, the subclass, is not a named class.

Lastly we test if $(\oent{Frank}, \oent{Susan}) : \oent{eats}$.
The call to the corresponding algorithm is
\[ \Call{testObjectPropertyAssertion}{\oent{eats}, \oent{Frank}, \oent{Susan}} \]
This first checks if $\oent{Susan}$ is one of the inferred object property values which $\oent{Frank}$ $\oent{eats}$.
It is not, so we enter the for-loop.
$N$ takes the named classes of which $\oent{Susan}$ is an inferred instance, which are $\{\oent{Giraffe}, \oent{Mammal}, \oent{Animal}\}$.
For each of these, we check if $\oent{Frank}$ is an instance of $\neg \exists \oent{eats} . N$.
When $N$ is $\oent{Animal}$, this is true, so the algorithm returns inconsistent.

% Alternatives:
% \[ \Call{testClassAssertion}{\oent{Plant}, \oent{Susan}} \]
% \[ \Call{testClassAssertion}{\exists \oent{eats}^- .\, \oent{Herbivore}, \oent{Susan}} \]
% \[ \Call{ObjectPropertyRange}{\oent{eats}, \oent{Animal}} \]

\end{document}
