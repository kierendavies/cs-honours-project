% !TEX root = paper.tex
\documentclass[paper.tex]{subfiles}

\begin{document}

\subsection{Class axioms}
\label{sec:algorithms:class}

In the class axioms permitted by OWL 2, all arguments may be arbitrary class expressions, not only named classes, except $\oaxiom{DisjointUnion}(N, C_1, \ldots, C_n)$ in which $N$ must be a named class.  Consequently, to determine if $C \sqsubseteq D$ it is not sufficient to check if $C \in \Call{getSubClasses}{D}$, because $C$ will not occur in this set if it is not a named class.  To resolve this, we build class expressions from the arguments and query them for satisfiability and instances.

Algorithm \ref{alg:testSubClassOf} tests subsumption of class expressions.

\begin{algorithm}[H]
  \caption{test $C \sqsubseteq D$}
  \label{alg:testSubClassOf}
  \begin{algorithmic}[1]
    \raggedright
    \Input{$C, D$ class expressions}
    \Function{testSubClassOf}{$C, D$}
      \If{$\Call{getInstances}{C \sqcap \neg D} \ne \emptyset$}
        \label{alg:testSubClassOf:checkInconsistent}
        \State \Return $\inconsistent$
        \label{alg:testSubClassOf:returnInconsistent}
      \ElsIf{$\Call{getSubClasses}{C \sqcap \neg D} \ne \emptyset$}
        \label{alg:testSubClassOf:checkIncoherent}
        \State \Return $\incoherent$
        \label{alg:testSubClassOf:returnIncoherent}
      \ElsIf{\Call{isSatisfiable}{$C \sqcap \neg D$}}
        \label{alg:testSubClassOf:checkAbsent}
        \State \Return $\absent$
      \Else
        \State \Return $\entailed$
        \label{alg:testSubClassOf:returnEntailed}
      \EndIf
    \EndFunction
  \end{algorithmic}
\end{algorithm}

\begin{lemma}
  \label{lem:subclassEntailsUnsat}
  For any set of axioms $O$ and class expressions $C$ and $D$,
  \[ O \:\sementails\: C \sqsubseteq D \;\iff\; O \:\sementails\: C \sqcap \neg D \sqsubseteq \bot \]
\end{lemma}
\begin{proof}
  For every interpretation $\I \synentails O$,
  \begin{align*}
    & O \:\sementails\: C \sqsubseteq D \\
    \iff{}& \I \:\synentails\: C \sqsubseteq D \\
    % \iff{}& C^\I \subseteq D^\I \\
    % \iff{}& (\forall x \in \Delta^\I) \; (x \in C^\I \!\!\implies\!\! x \in D^\I) \\
    % \iff{}& (\forall x \in \Delta^\I) \;\; x \notin C^\I \lor x \in D^\I \\
    % \iff{}& (\forall x \in \Delta^\I) \;\; \neg (x \in C^\I \land x \notin D^\I) \\
    % \iff{}& (\forall x \in \Delta^\I) \;\; \neg (x \in C^\I \land x \in \Delta^\I \setminus D^\I) \\
    % \iff{}& (\forall x \in \Delta^\I) \;\; x \notin C^\I \intersect (\Delta^\I \setminus D^\I) \\
    % \iff{}& (\forall x \in \Delta^\I) \;\; x \notin C^\I \intersect (\neg D)^\I \\
    % \iff{}& (\forall x \in \Delta^\I) \;\; x \notin (C \sqcap \neg D)^\I \\
    \iff{}& (C \sqcap \neg D)^\I = \emptyset = \bot^\I \\
    \iff{}& \I \:\synentails\: C \sqcap \neg D \sqsubseteq \bot \\
    \iff{}& O \:\sementails\: C \sqcap \neg D \sqsubseteq \bot
  \end{align*}
  The first and last bi-implications hold because we consider every possible interpretation.
\end{proof}

\begin{proposition}
  \label{prop:testSubClassOfEntailedSound}
  \textproc{testSubClassOf} is sound with respect to entailment.  That is,
  \begin{multline*}
    \Call{testSubClassOf}{C, D} = \entailed \implies \\
    \test_O(C \sqsubseteq D) = \entailed
  \end{multline*}
\end{proposition}
\begin{proof}
  The algorithm can only return entailed at line \ref{alg:testSubClassOf:returnEntailed}, so the three if-conditions must all be false.  So
  \begin{multline}
    \label{eqn:testSubClassOfEntailed}
    \Call{getInstances}{C \sqcap \neg D} = \emptyset \\
    {} \land \Call{getSubClasses}{C \sqcap \neg D} = \emptyset \\
    {} \land \lnot \Call{isSatisfiable}{C \sqcap \neg D}
  \end{multline}

  Now suppose $O \nsementails C \sqsubseteq D$.  By lemma \ref{lem:subclassEntailsUnsat}, $O \nsementails C \sqcap \neg D \sqsubseteq \bot$.  In other words, $C \sqcap \neg D$ is satisfiable, which contradicts the last term of equation \ref{eqn:testSubClassOfEntailed}.  Hence the supposition is false, so
  \begin{align*}
    & O \sementails C \sqsubseteq D \\
    \iff{}& \test_O(C \sqsubseteq D) = \entailed
    \qed
  \end{align*}
\end{proof}

\begin{proposition}
  \label{prop:testSubClassOfEntailedComplete}
  \textproc{testSubClassOf} is complete with respect to entailment.  That is,
  \begin{multline*}
    \test_O(C \sqsubseteq D) = \entailed \implies \\
    \Call{testSubClassOf}{C, D} = \entailed
  \end{multline*}
\end{proposition}
\begin{proof}
  As indicated in the proof of proposition \ref{prop:testSubClassOfEntailedSound}, the algorithm returns entailed if equation \ref{eqn:testSubClassOfEntailed} holds.

  From $\test_O(C \sqsubseteq D) = \entailed$ we have that $O \sementails C \sqsubseteq D$, and by lemma \ref{lem:subclassEntailsUnsat}, $O \sementails C \sqcap \neg D \sqsubseteq \bot$ so the last term of the equation is true.

  Since $C \sqcap \neg D$ is unsatisfiable, by the coherence precondition it has no named subclasses, and by the consistency precondition it has no instances.  Therefore the first and second terms of the equation are also true.

  Therefore equation \ref{eqn:testSubClassOfEntailed} holds, and so the algorithm returns entailed.
\end{proof}

\begin{proposition}
  \label{prop:testSubClassOfInconsistentSound}
  \textproc{testSubClassOf} is sound with respect to inconsistency.
\end{proposition}
\begin{proof}
  The algorithm can only return inconsistent at line \ref{alg:testSubClassOf:returnInconsistent}, so the first if-condition holds, so
  \[ \Call{getInstances}{C \sqcap \neg D} \ne \emptyset \]
  which means there exists an individual $a$ such that
  \begin{align*}
    & a : C \sqcap \neg D \\
    \iff{}& a : C \;\land\; a : \neg D
  \end{align*}

  Under $O \union (C \sqsubseteq D)$ it follows also that $a : D$, which is a contradiction, so
  \begin{align*}
    & O \union (C \sqsubseteq D) \sementails \bot \\
    \iff{}& \test_O(C \sqsubseteq D) = \inconsistent
    \qed
  \end{align*}
\end{proof}

\begin{proposition}
  \label{prop:testSubClassOfInconsistentComplete}
  \textproc{testSubClassOf} is complete with respect to inconsistency.
\end{proposition}
\begin{proof}
  We have that $O$ is consistent, so it has an interpretation, but $O \union (C \sqsubseteq D)$ is inconsistent, so it has no interpretations.

  Suppose the algorithm does not return inconsistent.  Then it must be that
  \[ \Call{getInstances}{C \sqcap \neg D} = \emptyset \]
  But in this case there exists an interpretation $\I$ which models both $O$ and $O \union (C \sqsubseteq D)$.  Let $\I$ be the interpretation of $O$ with the smallest domain.  This means that the interpretation of any class $E$ only contains elements which correspond to individuals which must be in that class.
  \[ E^\I = \{ x \in \Delta^\I \mid (\exists a \in \signatureI(O)) \; a : E \land a^\I = x \} \]
  This clearly still models $O$ because every individual is still in all classes it is entailed to be in.

  Under the supposition, we have that
  \[ (C \sqcap \neg D)^\I = \emptyset \]
  So for any individual $a$,
  \[ a : \neg (C \sqcap \neg D) \]
  Letting $a : C$,
  \begin{align*}
    a : \neg (C \sqcap \neg D)
    \implies{}& a : \neg C \sqcup D \\
    \implies{}& a : D
  \end{align*}
  From the construction of $\I$, this means that
  \begin{align*}
    & C^\I \subseteq D^\I \\
    \implies{}& \I \synentails C \sqsubseteq D
  \end{align*}
  So $\I$ also models $O \union (C \sqsubseteq D)$.

  This contradicts the initial condition that $O \union (C \sqsubseteq D)$ is inconsistent, so the supposition must be false, and therefore the algorithm returns inconsistent.
\end{proof}

\begin{proposition}
  \label{prop:testSubClassOfIncoherentSound}
  \textproc{testSubClassOf} is sound with respect to incoherence.
\end{proposition}
\begin{proof}
  The algorithm can only return incoherent at line \ref{alg:testSubClassOf:returnIncoherent}, so the first if-condition must be false and the second true.  So
  \begin{multline*}
    \Call{getInstances}{C \sqcap \neg D} = \emptyset \\
    {} \land \Call{getSubClasses}{C \sqcap \neg D} \ne \emptyset
  \end{multline*}
  Therefore, by the second term, there exists some named class $N \in \signatureC(O)$ such that
  \[ N \sqsubseteq C \sqcap \neg D \]
  By the contrapositive of proposition \ref{prop:testSubClassOfInconsistentComplete}, $O \union (C \sqsubseteq D)$ is consistent, so by lemma \ref{lem:subclassEntailsUnsat},
  \begin{align*}
    & O \union (C \sqsubseteq D) \:\sementails\: C \sqcap \neg D \sqsubseteq \bot \\
    \implies{}& O \union (C \sqsubseteq D) \:\sementails\: N \sqsubseteq \bot \\
    \implies{}& \test_O(C \sqsubseteq D) = \incoherent
    \qed
  \end{align*}
\end{proof}

\begin{proposition}
  \label{prop:testSubClassOfIncoherentComplete}
  \textproc{testSubClassOf} is complete with respect to incoherence.
\end{proposition}
\begin{proof}
  If the axiom $C \sqsubseteq D$ is added to $O$, then the only classes which are affected are $C$ and its subclasses.  Consider a named class $N \sqsubseteq C$.  If $O \nsementails N \sqsubseteq \neg D$ then it is possible that any element in the $N^\I$ is also in $D^\I$ and thus it is possible that $N \sqsubseteq D$.  If this is true for all such $N$, then they are all satisfiable in $O \union (C \sqsubseteq D)$ which is therefore coherent, so it must not be true for at least one $N$.  That is,
  \begin{align*}
    & (\exists N \in \signatureC(O)) \;\; N \sqsubseteq C \land N \sqsubseteq \neg D \\
    \iff{}& (\exists N \in \signatureC(O)) \;\; N \sqsubseteq C \sqcap \neg D \\
    \iff{}& (\exists N \in \signatureC(O)) \;\; N \in \Call{getSubClasses}{C \sqcap \neg D} \\
    \iff{}& \Call{getSubClasses}{C \sqcap \neg D} \ne \emptyset
  \end{align*}

  From the contrapositive of proposition \ref{prop:testSubClassOfInconsistentComplete} we have that the first if-condition is false, and we have shown that the second if-condition is true, so the algorithm returns incoherent.
\end{proof}

\begin{theorem}
  \label{thm:testSubClassOf}
  \textproc{testSubClassOf} is correct and terminating.
\end{theorem}
\begin{proof}
  It has been shown that the algorithm is sound and complete with respect to entailment, inconsistency, and incoherence.  By definition, the result is absent when it is not one of these other three.  Therefore the algorithm returns the correct result in all cases.

  Termination is trivial, since the algorithm contains no loops or recursion.
\end{proof}

Algorithm \ref{alg:testEquivalentClasses} tests equivalence if class expressions.
In the algorithm we use the integer variables $i$ and $j$ to iterate over the indices of the class expressions given as arguments, and the variables $r$ and $r'$ to store intermediate test results.

\begin{algorithm}[H]
  \caption{test $C_1 \equiv \ldots \equiv C_n$}
  \label{alg:testEquivalentClasses}
  \begin{algorithmic}[1]
    \raggedright
    \Input{
      $C_1, \ldots, C_n$ class expressions \\
      $n \ge 2$
    }
    \Function{testEquivalentClasses}{$C_1, \ldots, C_n$}
      \State $r \gets \entailed$
      \For{$i \gets 1$ \To $n$}
        \For{$j \gets 1$ \To $n$}
          \State $r' \gets \Call{testSubClassOf}{C_i, C_j}$
          \label{alg:testEquivalentClasses:inner}
          \State $r \gets \max(r, r')$
        \EndFor
      \EndFor
      \State \Return $r$
    \EndFunction
  \end{algorithmic}
\end{algorithm}

\begin{theorem}
  \sloppy~
  \textproc{testEquivalentClasses} is correct and terminating.
\end{theorem}
\begin{proof}
  By definition, $C_i \equiv C_j$ if and only if $C_i \sqsubseteq C_j$ and $C_j \sqsubseteq C_i$.  It is immediately clear from this that the algorithm is sound with respect to inconsistency and incoherence, and sound and complete with respect to entailment.  It remains to be shown that it is complete with respect to inconsistency and incoherence.

  Suppose the algorithm does not return inconsistent.  Then, for all $i,j \le n$
  \[ \Call{testSubClassOf}{C_i, C_j} \ne \inconsistent \]
  otherwise $r$ would take the value inconsistent because it is greater than other values.
  Therefore, for any individual $a$,
  \[ a : C_i \implies a : C_j \]
  So if $a$ is in any of $C_1, \ldots, C_n$ it is in all of them.  By the same construction as in the proof of proposition \ref{prop:testSubClassOfInconsistentComplete}, $O \union (C_1 \equiv \ldots \equiv C_n)$ has an interpretation and is not inconsistent.  Therefore, by contrapositive, the algorithm is complete with respect to inconsistency.

  Suppose the algorithm does not return incoherent.  If it returns inconsistent then by the above $O \union (C_1 \equiv \ldots \equiv C_n)$ is inconsistent.  If not, then it is consistent, and for all $i,j \le n$
  \[ \Call{getSubClasses}{C_i \sqcap \neg C_j} = \emptyset \]
  As in the proof of proposition \ref{prop:testSubClassOfIncoherentComplete}, there exists an interpretation $I \synentails O$ where, for any class expression $E$,
  \[ E^\I \ne \emptyset \iff (\exists N \in \signatureC(O)) \; N \sqsubseteq E \]
  So again under $O \union (C_1 \equiv \ldots \equiv C_n)$ we have
  \begin{align*}
    & (C_i \sqcap \neg C_j)^\I = \emptyset \\
    \implies{}& C_i^\I = (C_i \sqcap C_j)^\I \\
    \implies{}& \I \synentails C_i \sqsubseteq C_j
  \end{align*}
  This is true for all pairs, so
  \[ \I \synentails C_1 \equiv \ldots \equiv C_n \]
  So $\I$ also models $O \union (C_1 \equiv \ldots \equiv C_n)$ which is therefore coherent.  Therefore, by contrapositive, the algorithm is complete with respect to incoherence.

  All loops are bounded by $n$, which is finite, and all calls to \textproc{testSubClassOf} terminate, therefore the entire algorithm terminates.
\end{proof}

It is not sufficient to implement \textproc{testEquivalentClasses} by simply testing each of $C_2$ to $C_n$ for equivalence to $C_1$, as shown by a counterexample.  Let
\[ O = \{ N \equiv C_2 \sqcap \neg C_3 \} \]
It can be seen that $\Call{testSubClassOf}{C_2, C_3}$ returns incoherent, but this would not be reporting by testing only $C_1 \sqsubseteq C_2$, $C_2 \sqsubseteq C_1$, $C_1 \sqsubseteq C_3$, and $C_3 \sqsubseteq C_1$.

Algorithm \ref{alg:testDisjointClasses} tests pairwise disjointness of class expressions.

\begin{algorithm}[H]
  \caption{test $\oaxiom{DisjointClasses}(C_1, \ldots, C_n)$}
  \label{alg:testDisjointClasses}
  \begin{algorithmic}[1]
    \raggedright
    \Input{
      $C_1, \ldots, C_n$ class expressions \\
      $n \ge 2$
    }
    \Function{testDisjointClasses}{$C_1, \ldots, C_n$}
      \State $r \gets \entailed$
      \For{$i \gets 1$ \To $n-1$}
        \For{$j \gets i+1$ \To $n$}
          \State $r' \gets \Call{testSubClassOf}{C_i, \neg C_j}$
          \label{alg:testDisjointClasses:inner}
          \State $r \gets \max(r, r')$
        \EndFor
      \EndFor
      \State \Return $r$
    \EndFunction
  \end{algorithmic}
\end{algorithm}

\begin{proposition}
  \sloppy~
  \textproc{testDisjointClasses} is sound and complete with respect to entailment.
\end{proposition}
\begin{proof}
  The algorithm returns entailed if and only if the call to \textproc{testSubClassOf} at line \ref{alg:testDisjointClasses:inner} return entailed for every iteration of the enclosing.  If it did not, then $r'$ would have a greater value which would then be bound to $r$ and eventually returned.  Formally, for all $C_i$ and $C_j$ in $\{C_1, \ldots, C_n\}$ with $i < j$,
  \begin{align*}
    & \Call{testSubClassOf}{C_i, \neg C_j} = \entailed \\
    \iff{}& \Call{getInstances}{C_i \sqcap C_j} \ne \emptyset \\
      & \qquad \land \Call{getSubClasses}{C_i \sqcap C_j} \ne \emptyset \\
      & \qquad\qquad \land \neg \Call{isSatisfiable}{C_i \sqcap C_j}
  \end{align*}
  If this is true then $C_i \sqcap C_j \sqsubseteq \bot$, which by definition makes $C_i$ and $C_j$ disjoint, so the axiom is entailed.  Conversely, if the axiom is entailed then $C_i \sqcap C_j \sqsubseteq \bot$, and with the preconditions of consistency and coherence we have that all three terms are true and thus the algorithm returns entailed.
\end{proof}

\begin{proposition}
  \label{prop:testDisjointClassesInconsistentSound}
  \textproc{testDisjointClasses} is sound with respect to inconsistency.
\end{proposition}
\begin{proof}
  The algorithm returns inconsistent if and only if $r'$ takes the value inconsistent for some iteration of the nested for-loops, since it is the greatest test result value.  So for some $C_i$ and $C_j$ with $i < j$
  \begin{align*}
    & \Call{testSubClassOf}{C_i, \neg C_j} = \inconsistent \\
    \iff{}& \Call{getInstances}{C_i \sqcap C_j} \ne \emptyset \\
    \iff{}& (\exists a \in \signatureI(O)) \;\; a : C_i \sqcap C_j
  \end{align*}
  But with the axiom added to the ontology, $C_i \sqcap C_j \sqsubseteq \bot$, thus $a : \bot$ which is a contradiction, therefore
  \[ \test_O(\oaxiom{DisjointClasses}(C_1, \ldots, C_n)) = \inconsistent \qed \]
\end{proof}

\begin{proposition}
  \label{prop:testDisjointClassesInconsistentComplete}
  \sloppy
  \textproc{testDisjointClasses} is complete with respect to inconsistency.
\end{proposition}
\begin{proof}
  Suppose the algorithm does not return inconsistent.  Then for all $C_i$ and $C_j$ with $i < j$
  \begin{align*}
    & \Call{testSubClassOf}{C_i, \neg C_j} \ne \inconsistent \\
    \iff{}& \Call{getInstances}{C_i \sqcap C_j} = \emptyset
  \end{align*}

  Since $C_i$ and $C_j$ being disjoint means only that $C_i \sqcap C_j$ is empty, and it does not have any instances, the new axiom does not contradict $O$.  Therefore $O$ with the axiom added has interpretations, such as the interpretation of $O$ with the smallest domain, and consequently it is consistent.  Thus, by contrapositive, the proposition holds.
\end{proof}

\begin{proposition}
  \sloppy~
  \textproc{testDisjointClasses} is sound with respect to incoherence.
\end{proposition}
\begin{proof}
  If the algorithm returns incoherent then, by the contrapositive of proposition \ref{prop:testDisjointClassesInconsistentComplete}, for all $C_i$ and $C_j$ with $i < j$
  \[ \Call{testSubClassOf}{C_i, \neg C_j} \ne \inconsistent \]
  But for some $C_i$ and $C_j$ it must be that
  \begin{align*}
    & \Call{testSubClassOf}{C_i, \neg C_j} = \incoherent \\
    \iff{}& \Call{getInstances}{C_i \sqcap C_j} = \emptyset \\
      & \qquad \land \Call{getSubClasses}{C_i \sqcap C_j} \ne \emptyset
  \end{align*}
  From the last term it follows that there exists a named class $N \sqsubseteq C_i \sqcap C_j$.  But if the axiom is added to $O$, this class is empty, resulting in incoherence.
\end{proof}

\begin{proposition}
  \sloppy~
  \textproc{testDisjointClasses} is complete with respect to incoherence.
\end{proposition}
\begin{proof}
  If $\oaxiom{DisjointClasses}(C_1, \ldots, C_n)$ is added to $O$, the only classes affected are the intersections $C_i \sqcap C_j$ and their subclasses, which become unsatisfiable.  If this results in incoherence, then there must have been a named class $N \sqsubseteq C_i \sqcap C_j$ for some arguments $C_i$ and $C_j$.  If any $N$ were not a subclass of such an intersection, then it would remain satisfiable with the addition of the axiom.

  Note that, by the consistency precondition, no $C_i \sqcap C_j$ has instances.

  Given this $C_i$ and $C_j$, without loss of generality let $i < j$.  When the for-loops reach the respective $i$ and $j$, we will have that
  \begin{align*}
    & \Call{getInstances}{C_i \sqcap C_j} = \emptyset \\
      & \qquad \land \Call{getSubClasses}{C_i \sqcap C_j} \ne \emptyset \\
    \iff{}& \Call{testSubClassOf}{C_i, \neg C_j} = \incoherent
  \end{align*}

  By the contrapositive of proposition \ref{prop:testDisjointClassesInconsistentSound}, none of the calls to \textproc{testSubClassOf} return inconsistent, and therefore the maximum value of $r$ which is eventually returned is incoherent.
\end{proof}

\begin{theorem}
  \textproc{testDisjointClasses} is correct and terminating.
\end{theorem}
\begin{proof}
  Correctness is shown as in the proof of theorem \ref{thm:testSubClassOf}.  The algorithm terminates because all loops are bounded by $n$ and all calls to \textproc{testSubClassOf} terminate.
\end{proof}

Algorithm \ref{alg:testDisjointUnion} tests the disjoint union axiom.

\begin{algorithm}[H]
  \caption{test $\oaxiom{DisjointUnion}(N, C_1, \ldots, C_n)$}
  \label{alg:testDisjointUnion}
  \begin{algorithmic}[1]
    \raggedright
    \Input{
      $N$ named class \\
      $C_1, \ldots, C_n$ class expressions \\
      $n \ge 2$
    }
    \Function{testDisjointUnion}{$N, C_1, \ldots, C_n$}
      \State $r_1 \gets \Call{testEquivalentClasses}{N, C_1 \sqcup \ldots \sqcup C_n}$
      \State $r_2 \gets \Call{testDisjointClasses}{C_1, \ldots, C_n}$
      \State \Return $\max(r_1, r_2)$
    \EndFunction
  \end{algorithmic}
\end{algorithm}

\begin{theorem}
  \textproc{testDisjointUnion} is correct and terminating.
\end{theorem}
\begin{proof}
  Correctness is clearly seen from the definition of $\oaxiom{DisjointUnion}$, which states separately that $N \equiv C_1 \sqcup \ldots \sqcup C_n$ and $\oaxiom{DisjointClasses}(C_1, \ldots, C_n)$.  Termination is trivial.
\end{proof}

\end{document}
