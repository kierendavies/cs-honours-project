% !TEX root = paper.tex
\documentclass[paper.tex]{subfiles}

\begin{document}

\subsection{Object property axioms}
\label{sec:algorithms:objprop}

We present all algorithms in this section without proof of correctness, as the OWL 2 specification \cite{W3C:OWL2Syntax} describes how the respective object property axioms can be written as class axioms.

Algorithms \ref{alg:testObjectPropertyDomain} and \ref{alg:testObjectPropertyRange} test that an object property expression has a given class expression as its domain and range, respectively.

\begin{algorithm}[H]
  \caption{test $\oaxiom{ObjectPropertyDomain}(R, C)$}
  \label{alg:testObjectPropertyDomain}
  \begin{algorithmic}[1]
    \raggedright
    \Input{
      $R$ object property expression \\
      $C$ class expression
    }
    \Function{testObjectPropertyDomain}{$R, C$}
      \State \Return \Call{testSubClassOf}{$\exists R . \top, C$}
    \EndFunction
  \end{algorithmic}
\end{algorithm}

\begin{algorithm}[H]
  \caption{test $\oaxiom{ObjectPropertyRange}(R, C)$}
  \label{alg:testObjectPropertyRange}
  \begin{algorithmic}[1]
    \raggedright
    \Input{
      $R$ object property expression \\
      $C$ class expression
    }
    \Function{testObjectPropertyRange}{$R, C$}
      \State \Return \Call{testSubClassOf}{$\top, \forall R . C$}
    \EndFunction
  \end{algorithmic}
\end{algorithm}

Algorithm \ref{alg:FunctionalObjectProperty} tests that an object property expression is functional, and algorithm \ref{alg:InverseFunctionalObjectProperty} that it is inverse-functional.

\begin{algorithm}[H]
  \caption{test $\oaxiom{FunctionalObjectProperty}(R)$}
  \label{alg:FunctionalObjectProperty}
  \begin{algorithmic}[1]
    \raggedright
    \Input{
      $R$ object property expression
    }
    \Function{FunctionalObjectProperty}{$R$}
      \State \Return \Call{testSubClassOf}{$\top, \oaxiom{ObjectMaxCardinality}(1, R)$}
    \EndFunction
  \end{algorithmic}
\end{algorithm}

\begin{algorithm}[H]
  \caption{test $\oaxiom{InverseFunctionalObjectProperty}(R)$}
  \label{alg:InverseFunctionalObjectProperty}
  \begin{algorithmic}[1]
    \raggedright
    \Input{
      $R$ object property expression
    }
    \Function{InverseFunctionalObjectProperty}{$R$}
      \State \Return \Call{testSubClassOf}{$\top, \oaxiom{ObjectMaxCardinality}(1, R^-)$}
    \EndFunction
  \end{algorithmic}
\end{algorithm}

Algorithm \ref{alg:ReflexiveObjectProperty} tests that an object property expression is reflexive, and algorithm \ref{alg:IrreflexiveObjectProperty} that it is irreflexive.

\begin{algorithm}[H]
  \caption{test $\oaxiom{ReflexiveObjectProperty}(R)$}
  \label{alg:ReflexiveObjectProperty}
  \begin{algorithmic}[1]
    \raggedright
    \Input{
      $R$ object property expression
    }
    \Function{ReflexiveObjectProperty}{$R$}
      \State \Return \Call{testSubClassOf}{$\top, \oaxiom{ObjectHasSelf}(R)$}
    \EndFunction
  \end{algorithmic}
\end{algorithm}

\begin{algorithm}[H]
  \caption{test $\oaxiom{IrreflexiveObjectProperty}(R)$}
  \label{alg:IrreflexiveObjectProperty}
  \begin{algorithmic}[1]
    \raggedright
    \Input{
      $R$ object property expression
    }
    \Function{IrreflexiveObjectProperty}{$R$}
      \State \Return \Call{testSubClassOf}{$\oaxiom{ObjectHasSelf}(R), \bot$}
    \EndFunction
  \end{algorithmic}
\end{algorithm}

\end{document}
