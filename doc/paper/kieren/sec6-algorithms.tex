% !TEX root = paper.tex
\documentclass[paper.tex]{subfiles}

\begin{document}

\section{Algorithms and analysis}
\label{sec:algorithms}

We now present the algorithms and analysis, in the context of an ontology $O$.  We base the coverage on OWL 2, but due to constraints we only cover class axioms (TBox) and assertions (ABox).  This fully covers $\mathcal{ALC}$, which accounts for the majority of real use cases.  \todo[fix bullshit]

As justified in section \ref{sec:model}, we assume the following preconditions:
\begin{itemize}[nosep]
  \item The ontology under test $O$ is consistent and coherent.
  \item The axiom under test contains only entities which are declared in the ontology.
\end{itemize}

Each algorithm is named according to the axiom it tests, as written in OWL 2 functional syntax, prepended with ``\textproc{test}''.  For example, the algorithm for testing $\oaxiom{SubClassOf}$ axioms is named \textproc{testSubClassOf}.  We use the variables $C$, $D$ for class expressions; $N$ for a named class; $a$, $b$ for individuals; and $R$, $S$ for object properties.  \todo[move]

We address class axioms in section \ref{sec:algorithms:class}, assertions in section \ref{sec:algorithms:assert}, and object property axioms in section \ref{sec:algorithms:objprop}.  \todo[check]
% We do not address data property axioms or annotation axioms because their grammar and semantics are a equivalent to a fragment of object properties, and they are seldom involved in reasoning.
% We also do not address $\oaxiom{HasKey}$ axioms because they can be regarded as a mistaken addition to OWL 2 \cite{Keet:Personal}.

\subfile{sec6.1-class}
\subfile{sec6.2-assert}
\subfile{sec6.3-objprop}

\end{document}
