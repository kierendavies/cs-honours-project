\documentclass[final,fleqn]{beamer}
\usepackage[
  size=a1,
  orientation=portrait,
  scale=1.5,
]{beamerposter}
\usetheme{poster}
\epstopdfsetup{outdir=./}

% \usepackage[english]{babel}
% \usepackage[english]{isodate}
% \cleanlookdateon

\usepackage{ragged2e}
\usepackage{mathtools}
\usepackage{tikz}
\usetikzlibrary{arrows.meta}
\usetikzlibrary{patterns}

\pgfdeclarepatternformonly{unsatisfiable}{\pgfqpoint{-1pt}{-1pt}}{\pgfqpoint{37pt}{37pt}}{\pgfqpoint{36pt}{36pt}}{
  \pgfsetlinewidth{3pt}
  \pgfpathmoveto{\pgfqpoint{0pt}{0pt}}
  \pgfpathlineto{\pgfqpoint{36.1pt}{36.1pt}}
  \pgfusepath{stroke}
}

\title{Towards\\Test-Driven Development\\of Ontologies}
\author{Kieren Davies\hspace{0.5em}\nolinkurl{<kieren@kdavi.es>}}
\supervisor{Maria Keet\hspace{0.5em}\nolinkurl{<mkeet@cs.uct.ac.za>}}
\institute{%
  Department of Computer Science \\
  University of Cape Town
}
\website{\nolinkurl{https://people.cs.uct.ac.za/~dvskie001/}}

\begin{document}
\begin{frame}{}

\begin{columns}[onlytextwidth]
  \begin{column}{.45\linewidth}
    \begin{minipage}[t][.65\paperheight][s]{\columnwidth}
      \begin{block}{Ontologies and TDD}
        \justifying
        \setlength{\parskip}{1ex}\vspace{-1ex}
        An ontology is a formal, machine-readable representation of knowledge about entities, used with automated reasoning.

        Test-Driven Development (TDD) is a software engineering methodology with two rules:
        \vspace{-1ex}
        \begin{itemize}
          \item Write new code only if an automated test has failed,
          \item Eliminate duplication.
        \end{itemize}

        To apply TDD to ontology engineering, we need sophisiticated tools.
        Current tools do not support all complex axioms, and only report pass, fail, or missing entity.
      \end{block}
      \vfill
      \begin{block}{Use cases\phantom{y}}
        \begin{enumerate}
          \item Permanent test suite: demonstrate quality or detect regressions.
          \item Temporary tests: explore inferences or predict consequences of a change.
        \end{enumerate}
      \end{block}
      \vfill
      \begin{block}{Objectives}
        Support development of ontology testing tools by describing new algorithms, which
        \begin{itemize}
          \item Allow testing of any possible axiom,
          \item Use fast reasoner queries,
          \item Report detailed results,
          \item Are proven to be correct.
        \end{itemize}
      \end{block}
      \vfill
      \begin{block}{Formal model\phantom{y}}
        % \justifying
        \setlength{\parskip}{1ex}\vspace{-1ex}
        Given an ontology $O$ which is consistent (no contradictions) and coherent (all classes satisfiable), and an axiom $A$ containing only entities from $O$.

        To test, determine if $A$ is already entailed by $O$, would make it inconsistent or incoherent, or none of these.
        \vspace{1ex}
        \[
          \mathrm{test}_O(A) =
          \begin{cases}
            \text{entailed} &
              \text{if } O \vdash A \\
            \text{inconsistent} &
              \text{if } O \cup A \vdash \bot \\
            \text{incoherent} &
              \text{if } (\exists C)\; O \cup A \vdash C \sqsubseteq \bot \\
            \text{absent} &
              \text {otherwise} \\
          \end{cases}
        \]
      \end{block}
    \end{minipage}
  \end{column}
  \begin{column}{.45\linewidth}
    \begin{minipage}[t][.65\paperheight][s]{\columnwidth}
      \begin{block}{Algorithm example}
        \begin{center}
          To test a subclass axiom $C \sqsubseteq D$, \\
          call $\textsc{testSubClassOf}(C, D)$
        \end{center}

        \vspace{1cm}
        \hspace{\dimexpr.5\linewidth-6.5cm}
        \begin{tikzpicture}[line width=2pt]
          \draw[fill=prill!30,rounded corners=2cm] (0, 0) rectangle (13, 10);
          \fill[fill=sec,fill opacity=0.5] (4.5, 5) circle[radius=3.5];
          \draw (4.5, 7) node{$C$};
          \draw[fill=ter,fill opacity=0.5,text opacity=1] (9.5, 5) circle[radius=2.5] node{$D$};
          \draw (4.5, 5) circle[radius=3.5];

          \coordinate (a) at (5, 3.5);
          \filldraw (a) circle[radius=4pt] node[right]{$a$};
          \draw[fill=sec!80] (3, 5) circle[radius=1] node(N){$N$};

          \draw[-{Latex[length=5mm]}, dashed] (a) .. controls (0, 2) and (0.5, -1) .. (2, -1);
          \node[right,align=left] at (2, -1.5) {
            1. $C \sqcap \neg D$ has instances? \\
            \hspace{1em}$\rightarrow$ \textbf{inconsistent}
          };
          \draw[-{Latex[length=5mm]}, dashed] (N) .. controls (0, 3) and (0, -3.5) .. (2, -3.5);
          \node[right,align=left] at (2, -4) {
            2. $C \sqcap \neg D$ has named subclasses? \\
            \hspace{1em}$\rightarrow$ \textbf{incoherent}
          };
        \end{tikzpicture}

        \vspace{1cm}
        \hspace{\dimexpr.5\linewidth-6.5cm}
        \begin{tikzpicture}[line width=2pt]
          \draw[fill=prill!30,rounded corners=2cm] (0, 0) rectangle (13, 10);
          \fill[fill=sec,fill opacity=0.5] (4.5, 5) circle[radius=3];
          \begin{scope}[even odd rule]
            \clip (8.5, 5) circle[radius=3] (0, 0) rectangle (13, 10);
            \fill[pattern=unsatisfiable,pattern color=red] (4.5, 5) circle[radius=3];
          \end{scope}
          \draw[fill=ter,fill opacity=0.5] (8.5, 5) circle[radius=3];
          \node at (9, 5) {$D$};
          \draw (4.5, 5) circle[radius=3];
          \node at (4, 5) {$C$};

          \node[right,align=left] at (2, -1.5) {
            3. $C \sqcap \neg D$ unsatisfiable? \\
            \hspace{1em}$\rightarrow$ \textbf{entailed}
          };
          % \node[align=left] at (2, -4) {\phantom{X}\\\phantom{X}};
          \node[right,align=left] at (2, -4.5) {
            4. None of the above? \\
            \hspace{1em}$\rightarrow$ \textbf{absent}
          };
        \end{tikzpicture}
      \end{block}
      \vfill
      \begin{block}{Conclusion\phantom{y}}
        \begin{itemize}
          \item Novel formal model of ontology testing.
          \vspace{1ex}
          \item New algorithms satisfy use cases and objectives.
          \vspace{1ex}
          \item Model used to prove correctness of algorithms.
          \vspace{1ex}
          \item Algorithms can easily be implemented in new or existing tools.
        \end{itemize}
      \end{block}
      % \vfill
      % \begin{block}{}
      %   Benchmarking with different reasoners to be completed by Ameerah Allie.
      % \end{block}
      \vspace{2cm}
    \end{minipage}
  \end{column}
\end{columns}

\end{frame}
\end{document}
